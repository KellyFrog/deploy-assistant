%%%%%%%% ICML 2025 EXAMPLE LATEX SUBMISSION FILE %%%%%%%%%%%%%%%%%

\documentclass{article}

% 项目要求
% 1. 基础功能层(必须完成)
% –  集成至少1个大语言模型API(如GPT-3.5/4、qwen、deepseek等),展示不同参数(如temperature=0.7 vs 1.2)对输出的影响对比
% –  实现核心功能闭环,需定义明确的输入输出规范(如JSON Schema)并实现校验机制
% –  包含基础交互界面(命令行/CUI/GUI任选)
% –  处理一些真实场景测试数据作为展示
% –  包含下方“技术实现规范”中的要点。
% 2. 进阶功能层(选择性完成)
% –  实现多模态扩展(图像/语音输入输出)
% –  实现高质量结构化输出控制,能95%的概率生成符合要求的格式
% –  构建领域知识库增强,如RAG架构需说明embedding模型选型(如text2vec)和检索策略(如FAISS索引)
% –  开发记忆机制(对话历史/用户画像)
% –  集成外部工具(包括但不限于计算器、数据库、API、Python代码解释器)
% –  实现一个自主“Agent”,需实现至少3种工具调用决策逻辑
% –  其他可能的额外功能
% 3. 创新维度(选择性完成)
% –  解决未被主流产品覆盖的需求痛点
% –  提出新颖的prompt engineering方案
% –  设计独特的输出呈现形式
% –  其他可能的创新形式
%    注:以上括号内的,均是举例,而非必须和仅仅。比如集成的外部工具可以从括号中选几个,也可以自己再额外设计。

% 技术实现规范
% 1. 模型调用
% –  必须展示API调用参数调优过程(temperature/top_p等)
% –  需处理流式响应,实现逐字/分块输出效果(如ChatGPT式打字机效果),禁用单次完整响应
% –  实现输入预处理(敏感词过滤/指令注入防护)
% 2. 系统架构
% –  需提供架构设计图(数据流图/模块关系图)
% –  要求模块化开发(至少3个独立功能模块)
% –  必须包含异常处理机制(API失败重试、结构化输出失败等)

% Recommended, but optional, packages for figures and better typesetting:
\usepackage{microtype}
\usepackage{graphicx}
\usepackage{subfigure}
\usepackage{booktabs} % for professional tables
\usepackage{ctex}

% hyperref makes hyperlinks in the resulting PDF.
% If your build breaks (sometimes temporarily if a hyperlink spans a page)
% please comment out the following usepackage line and replace
% \usepackage{icml2025} with \usepackage[nohyperref]{icml2025} above.
\usepackage{hyperref}



% Attempt to make hyperref and algorithmic work together better:
\newcommand{\theHalgorithm}{\arabic{algorithm}}

% Use the following line for the initial blind version submitted for review:
% \usepackage{icml2025}

% If accepted, instead use the following line for the camera-ready submission:
\usepackage[accepted]{icml2025}

% For theorems and such
\usepackage{amsmath}
\usepackage{amssymb}
\usepackage{mathtools}
\usepackage{amsthm}

% if you use cleveref..
\usepackage[capitalize,noabbrev]{cleveref}

%%%%%%%%%%%%%%%%%%%%%%%%%%%%%%%%
% THEOREMS
%%%%%%%%%%%%%%%%%%%%%%%%%%%%%%%%
\theoremstyle{plain}
\newtheorem{theorem}{Theorem}[section]
\newtheorem{proposition}[theorem]{Proposition}
\newtheorem{lemma}[theorem]{Lemma}
\newtheorem{corollary}[theorem]{Corollary}
\theoremstyle{definition}
\newtheorem{definition}[theorem]{Definition}
\newtheorem{assumption}[theorem]{Assumption}
\theoremstyle{remark}
\newtheorem{remark}[theorem]{Remark}

% Todonotes is useful during development; simply uncomment the next line
%    and comment out the line below the next line to turn off comments
%\usepackage[disable,textsize=tiny]{todonotes}
\usepackage[textsize=tiny]{todonotes}


% The \icmltitle you define below is probably too long as a header.
% Therefore, a short form for the running title is supplied here:
\icmltitlerunning{Submission and Formatting Instructions for ICML 2025}

\begin{document}

\twocolumn[
\icmltitle{人工智能基础 大作业报告}

%示例,根据自己的背景更改
\begin{icmlauthorlist}
\icmlauthor{陈凯丰}{2400934023}
\icmlauthor{队员姓名2}{背景1}
\icmlauthor{队员姓名3}{背景1}
\icmlauthor{队员姓名4}{背景2}
\icmlauthor{队员姓名5}{背景2}
\end{icmlauthorlist}


%示例,根据自己的背景更改
% \icmlaffiliation{背景1}{计算机学院, 北京大学, 年级}
% \icmlaffiliation{背景2}{环境学院, 北京大学, 年级}

% 显目关键词,根据自己的项目更改
\icmlkeywords{大模型,机器学习}

\vskip 0.3in
]

% 项目摘要
\begin{abstract}
很短的项目摘要
\end{abstract}

% 项目内容
\section{主题}

项目的主题和需求来自朱汶宣同学安装 Jupiter Notebook 的经历,由于一些依赖和 WSL 的问题,他希望能够有一款轻量级的产品能够帮助他定制化地精细分析运行命令过程中的问题。

随后我们将项目主题定为了 LLM 集成的轻量级终端。

\subsection{需求分析}

为了帮助用户解决在命令行下的问题,首先应当监控命令行的输出,然后在接到用户指令之后将这些信息处理并调用 LLM 获得建议,最后将建议呈现在用户界面上,供用户选择采用。

我们决定将开发目标平台定在 windows 上,因为小组几位同学使用的都是 windows 电脑,并且 windows 原生的命令行生态并不成熟,相对于 Linux 和 Mac 缺少统一的管理器,遇到的环境和配置问题可能更多。

\subsection{技术选型}

为了能够在短时间内进行开发,我们小组决定使用 python 进行开发,优先采用已有的开源库来提高开发效率。

在查阅资料之后,我们决定使用 pywinpty 库来实现 windows 下和终端的交互,这个封装了系统调用,维护了一个伪终端(Pseudo Terminal)对象,使得程序可以和运行当中的命令行进行交互。

对于和 LLM 交互的部分,我们决定使用较为成熟的 openai 库来实现和远程 API 的交互。

由于实现一个 GUI 过于复杂,需要实现的终端渲染内容过多(如虚拟控制字符等),我们决定实现一个 CLI,直接利用 windows 已经实现好的终端应用(如 windows terminal 等)。

\subsubsection{总体架构}

项目总体包括如下模块:

\begin{itemize}
    \setlength{\itemsep}{0.1em}
    \setlength{\parskip}{0.1em}
    \item 模拟终端模块:封装和 shell 的交互。
    \item CLI 模块:实现用户交互,维护命令行历史内容。
    \item 记忆模块:实现和 LLM 交互的短期和长期记忆。
    \item LLM 模块:封装和 LLM 的交互。
    \item agent 模块:总结和处理信息,将上下文、记忆和问题交给 LLM 模块。
    \item 部署模块:针对 git 仓库生成部署计划
    \item 安全模块:离线检测命令安全性。
    \item utils 模块:工具模块。
\end{itemize}

\subsubsection{模块简介}

CLI 模块是整个程序的入口以及交互界面,其中维护了一个内建的模拟终端对象和 Agent 对象,CLI 负责捕获用户的输出,判断是否是询问指令,并且将输入交给模拟终端或 Agent,同时异步地监听终端的输出。

模拟终端模块实现了一个模拟终端类,在设定启动命令后异步地读取内建终端的输出,将输出异步地将输出回传给 CLI。

LLM 模块,封装了获取 API Key 已经和远端 API 的调用。

记忆模块封装了读取中期和长期记忆的过程,以及总结生成新记忆的 prompt 工程。

agent 模块封装了和记忆以及 LLM 的交互,并通过 prompt 工程实现格式化回复。

utils 模块实现了一些独立的工具和功能,如行内刷新输出等。

\subsection{实现细节}

\subsubsection{模拟终端和 CLI}

pywinpty 的伪终端支持向 stdin 写、从 stdout 阻塞地读。

windows 下的 shell(如 powershell)本身维护了一个缓冲区,接受用户输入的字符,包括可显示字符和不可见的控制字符,如 Ctrl-C、Backspace、左键等,并且实时向 stdout 输出带有控制台虚拟终端序列的 ANSI 字节流。

windows 下的终端,如 conhost、windows terminal、git bash 等均实现了控制台虚拟终端序列的控制和渲染,事实上每输入一个字符,windows 下的终端软件会接收到 shell 输出的 ANSI 字节流,要求软件重新渲染最后一行,以此达到动态输入的目的。因此直接捕获字符写入伪终端这可能导致大量重复的输出,为了解析这些输出需要较大的工作量。

为了解决这个问题,我们决定牺牲部分终端的便捷性,采用一次输入一行的方法,如此输入不会重复出现,虚拟终端序列的去除也较为方便,同时还能够直接利用 windows 下终端对 ANSI 字节流的支持。

因此启动 CLI 的主进程就是一个循环,阻塞地读取用户一行的输入,并且将其经过处理交给 Agent 或写入内建伪终端。

由于 shell 的输出时间并不规律,从内建伪终端的读取必须是异步的,模拟终端类中新建了一个进程进行读取和传回。在 CLI 中,一个回调函数被传递给模拟终端类,在读取到输入后将新输入的内容加到历史队列中并输出到用户界面上,其中数据结构的锁都由 python 内置库管理。

在用户输入调用 Agent 时,CLI 会将目前所有的历史记录和目录等信息传递给 Agent,Agent 经过处理以及和 LLM 的交互后返回一个命令列表,用户确认后在模拟终端中执行。

\subsubsection{Agent 实现}

\subsubsection{LLM 接口处理}

\subsubsection{Prompt 工程}

\subsubsection{安全检查}

\subsubsection{历史记忆与用户画像}

\subsubsection{API 参数调优}


\subsection{评估对比}

% 好像最近有个开源项目叫 fuck,大概就是 fuck 以下就可以修正上一条命令的错误,但那个是基于本地匹配规则的,这一点可以写一写。

\subsection{反思}


\bibliography{example_paper}
\bibliographystyle{icml2025}


%%%%%%%%%%%%%%%%%%%%%%%%%%%%%%%%%%%%%%%%%%%%%%%%%%%%%%%%%%%%%%%%%%%%%%%%%%%%%%%
%%%%%%%%%%%%%%%%%%%%%%%%%%%%%%%%%%%%%%%%%%%%%%%%%%%%%%%%%%%%%%%%%%%%%%%%%%%%%%%
% APPENDIX
%%%%%%%%%%%%%%%%%%%%%%%%%%%%%%%%%%%%%%%%%%%%%%%%%%%%%%%%%%%%%%%%%%%%%%%%%%%%%%%
%%%%%%%%%%%%%%%%%%%%%%%%%%%%%%%%%%%%%%%%%%%%%%%%%%%%%%%%%%%%%%%%%%%%%%%%%%%%%%%
\newpage
\appendix
\onecolumn
\section{附录}

可以将一些额外的内容放在这里
%%%%%%%%%%%%%%%%%%%%%%%%%%%%%%%%%%%%%%%%%%%%%%%%%%%%%%%%%%%%%%%%%%%%%%%%%%%%%%%
%%%%%%%%%%%%%%%%%%%%%%%%%%%%%%%%%%%%%%%%%%%%%%%%%%%%%%%%%%%%%%%%%%%%%%%%%%%%%%%


\end{document}

%This document was modified from the file originally made available by
% Pat Langley and Andrea Danyluk for ICML-2K. This version was created
% by Shui Jie in 2025

