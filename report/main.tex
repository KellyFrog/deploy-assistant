%%%%%%%% ICML 2025 EXAMPLE LATEX SUBMISSION FILE %%%%%%%%%%%%%%%%%

\documentclass{article}

% Recommended, but optional, packages for figures and better typesetting:
\usepackage{microtype}
\usepackage{graphicx}
\usepackage{subfigure}
\usepackage{booktabs} % for professional tables

% hyperref makes hyperlinks in the resulting PDF.
% If your build breaks (sometimes temporarily if a hyperlink spans a page)
% please comment out the following usepackage line and replace
% \usepackage{icml2025} with \usepackage[nohyperref]{icml2025} above.
\usepackage{hyperref}



% Attempt to make hyperref and algorithmic work together better:
\newcommand{\theHalgorithm}{\arabic{algorithm}}

% Use the following line for the initial blind version submitted for review:
% \usepackage{icml2025}

% If accepted, instead use the following line for the camera-ready submission:
\usepackage[accepted]{icml2025}

% For theorems and such
\usepackage{amsmath}
\usepackage{amssymb}
\usepackage{mathtools}
\usepackage{amsthm}

% if you use cleveref..
\usepackage[capitalize,noabbrev]{cleveref}

%%%%%%%%%%%%%%%%%%%%%%%%%%%%%%%%
% THEOREMS
%%%%%%%%%%%%%%%%%%%%%%%%%%%%%%%%
\theoremstyle{plain}
\newtheorem{theorem}{Theorem}[section]
\newtheorem{proposition}[theorem]{Proposition}
\newtheorem{lemma}[theorem]{Lemma}
\newtheorem{corollary}[theorem]{Corollary}
\theoremstyle{definition}
\newtheorem{definition}[theorem]{Definition}
\newtheorem{assumption}[theorem]{Assumption}
\theoremstyle{remark}
\newtheorem{remark}[theorem]{Remark}

% Todonotes is useful during development; simply uncomment the next line
%    and comment out the line below the next line to turn off comments
%\usepackage[disable,textsize=tiny]{todonotes}
\usepackage[textsize=tiny]{todonotes}


% The \icmltitle you define below is probably too long as a header.
% Therefore, a short form for the running title is supplied here:
\icmltitlerunning{Submission and Formatting Instructions for ICML 2025}

\begin{document}

\twocolumn[
\icmltitle{人工智能基础 大作业模板}

%示例,根据自己的背景更改
\begin{icmlauthorlist}
\icmlauthor{队员姓名1}{背景1}
\icmlauthor{队员姓名2}{背景1}
\icmlauthor{队员姓名3}{背景1}
\icmlauthor{队员姓名4}{背景2}
\icmlauthor{队员姓名5}{背景2}
\end{icmlauthorlist}


%示例,根据自己的背景更改
\icmlaffiliation{背景1}{计算机学院, 北京大学, 年级}
\icmlaffiliation{背景2}{环境学院, 北京大学, 年级}

% 显目关键词,根据自己的项目更改
\icmlkeywords{大模型,机器学习}

\vskip 0.3in
]

% 项目摘要
\begin{abstract}
很短的项目摘要
\end{abstract}

% 项目内容
\section{主题}

、项目要求
1. 基础功能层(必须完成)
–  集成至少1个大语言模型API(如GPT-3.5/4、qwen、deepseek等),展示不同参数(如temperature=0.7 vs 1.2)对输出的影响对比
–  实现核心功能闭环,需定义明确的输入输出规范(如JSON Schema)并实现校验机制
–  包含基础交互界面(命令行/CUI/GUI任选)
–  处理一些真实场景测试数据作为展示
–  包含下方“技术实现规范”中的要点。
2. 进阶功能层(选择性完成)
–  实现多模态扩展(图像/语音输入输出)
–  实现高质量结构化输出控制,能95%的概率生成符合要求的格式
–  构建领域知识库增强,如RAG架构需说明embedding模型选型(如text2vec)和检索策略(如FAISS索引)
–  开发记忆机制(对话历史/用户画像)
–  集成外部工具(包括但不限于计算器、数据库、API、Python代码解释器)
–  实现一个自主“Agent”,需实现至少3种工具调用决策逻辑
–  其他可能的额外功能
3. 创新维度(选择性完成)
–  解决未被主流产品覆盖的需求痛点
–  提出新颖的prompt engineering方案
–  设计独特的输出呈现形式
–  其他可能的创新形式
   注:以上括号内的,均是举例,而非必须和仅仅。比如集成的外部工具可以从括号中选几个,也可以自己再额外设计。

、技术实现规范
1. 模型调用
–  必须展示API调用参数调优过程(temperature/top_p等)
–  需处理流式响应,实现逐字/分块输出效果(如ChatGPT式打字机效果),禁用单次完整响应
–  实现输入预处理(敏感词过滤/指令注入防护)
2. 系统架构
–  需提供架构设计图(数据流图/模块关系图)
–  要求模块化开发(至少3个独立功能模块)
–  必须包含异常处理机制(API失败重试、结构化输出失败等)


\subsection{需求分析}


\subsection{技术选型}


\subsection{实现细节}


\subsection{评估对比}


\subsection{反思}


\bibliography{example_paper}
\bibliographystyle{icml2025}


%%%%%%%%%%%%%%%%%%%%%%%%%%%%%%%%%%%%%%%%%%%%%%%%%%%%%%%%%%%%%%%%%%%%%%%%%%%%%%%
%%%%%%%%%%%%%%%%%%%%%%%%%%%%%%%%%%%%%%%%%%%%%%%%%%%%%%%%%%%%%%%%%%%%%%%%%%%%%%%
% APPENDIX
%%%%%%%%%%%%%%%%%%%%%%%%%%%%%%%%%%%%%%%%%%%%%%%%%%%%%%%%%%%%%%%%%%%%%%%%%%%%%%%
%%%%%%%%%%%%%%%%%%%%%%%%%%%%%%%%%%%%%%%%%%%%%%%%%%%%%%%%%%%%%%%%%%%%%%%%%%%%%%%
\newpage
\appendix
\onecolumn
\section{附录}

可以将一些额外的内容放在这里
%%%%%%%%%%%%%%%%%%%%%%%%%%%%%%%%%%%%%%%%%%%%%%%%%%%%%%%%%%%%%%%%%%%%%%%%%%%%%%%
%%%%%%%%%%%%%%%%%%%%%%%%%%%%%%%%%%%%%%%%%%%%%%%%%%%%%%%%%%%%%%%%%%%%%%%%%%%%%%%


\end{document}

%This document was modified from the file originally made available by
% Pat Langley and Andrea Danyluk for ICML-2K. This version was created
% by Shui Jie in 2025

